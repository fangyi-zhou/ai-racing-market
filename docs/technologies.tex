\documentclass[a4paper]{article}

%% Language and font encodings
\usepackage[english]{babel}
\usepackage[utf8x]{inputenc}
\usepackage[T1]{fontenc}
\usepackage{graphicx}

%% Sets page size and margins
\usepackage[a4paper,top=1cm,bottom=2cm,left=3cm,right=3cm,marginparwidth=1.75cm]{geometry}

\usepackage{hyperref}

\title{Project Technology Report}
\author{Group 28: Robert Holland, Ruiao Hu, Fangyi Zhou}

\begin{document}
\maketitle

\section{Language/Framework Choice}

The back end of this app is made in \texttt{node.js}. The choice has been made
for the following reasons. There are abundant Javascript packages available for
different components of this project which eases the development process.

The front end of this app uses \texttt{Angular} framework.

\section{User Code Execution}

User code is executed by spawning new processes and executing the code
alongside the server. The communication is done by piping \texttt{stdin} and
\texttt{stdout} to the server so that the user do not have to spend time
dealing with I/O. However, this method has potential security issues as
malicious code may cause damages to the server or other computers. Running code
inside a docker sandbox is planned to replace the current method.

\section{Simulation and Rendering}

Simulation is done using the Javascript physics engine \texttt{p2.js}, it is
done on the server side.
After simulation has been done on the server side, information on all car
positions are packed and sent to the client side. The client side uses the
Javascript Rendering Engine \texttt{PixiJS} and renders the graphics using the
data transmitted by the server.

\section{Database}

\texttt{Mongodb} has been used for this application. The service has been
integrated to \texttt{Heroku} and is easy to use.

\section{Authentication}

JSON Web Tokens (\href{https://tools.ietf.org/html/rfc7519}{RFC 7519}) have
been used for authentication purposes, this allows authentication to be done in
a secure and compact fashion.

\section{Continuous Integration}

The project uses \texttt{Travis CI} for continuous integration. After each push
to \texttt{Github}. The continuous integration service will start a build and
run tests using \texttt{npm test} after the build.
\begin{figure}
  \includegraphics[width=\linewidth]{build}
  \label{fig:boat1}
\end{figure}

\section{Deployment}

The projects runs on \texttt{Heroku}. A webhook is set so that the latest
commit will be automatically deployed if it passes the continuous integration
test as stated in previous section.

\end{document}
